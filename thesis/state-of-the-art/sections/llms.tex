\section{Large language models (LLMs)}

\subsection{Introduction}

Language modeling is a long-standing research topic, dating back to the 1950s with Shnnon's application of information theory to human language, where he measures how well simple n-gram language models predict or compress natural language text \cite{6773263}. Since then, statistical language modeling became fundamental to many natural language understanding and generation tasks, ranging from speech recognition, machine translation, to information retrieval \cite{Jelinek1997StatisticalMF} \cite{Manning_Raghavan_Schütze_2008}.

Recent advances in transformer-based large language models (LLMs), pretrained on web-scale text corpora, have significantly enhanced the capabilities of these models. For instance, OpenAI's ChatGPT and GPT-4 are now used not only for natural language processing but also as general task solvers, exemplified by their integration into Microsoft's Co-Pilot systems. These models can follow complex human instructions and perform multi-step reasoning when necessary. Consequently, LLMs are becoming fundamental building blocks for the development of general-purpose AI agents or artificial general intelligence (AGI).

LLMs are large-scale, pre-trained, statistical language models based on neural networks. Their recent success is the result of decades of research and development in language modeling, which can be divided into four distinct waves, each with its own starting points and progression: statistical language models, neural language models, pre-trained language models, and LLMs.

Early neural language models (NLMs) \cite{NIPS2000_728f206c}, \cite{schwenk-etal-2006-continuous}, \cite{mikolov10_interspeech}, \cite{graves2014generating} address data sparsity by mapping words to low-dimensional continuous vectors, known as embedding vectors, and predicting the next word based on the aggregation of these vectors using neural networks. The embedding vectors learned by NLMs create a hidden space where the semantic similarity between vectors can be easily computed as their distance. This enables the computation of semantic similarity between any two inputs, regardless of their forms (e.g., queries versus documents in web search \cite{10.1145/2505515.2505665}, \cite{gao2022neural}, sentences in different languages in machine translation \cite{sutskever2014sequence}, \cite{cho2014properties}) or modalities (e.g., image and text in image captioning \cite{cho2014properties}, \cite{vinyals2015tell}). However, early NLMs are task-specific models, as they are trained on task-specific data, resulting in a hidden space that is also task-specific.

Pre-trained language models (PLMs), unlike early neural language models (NLMs), are task-agnostic. This generality also extends to the learned hidden embedding space. The training and inference of PLMs follow the pre-training and fine-tuning paradigm, where language models utilizing recurrent neural networks \cite{peters2018deep} or transformers \cite{devlin2019bert}, \cite{liu2019roberta}, \cite{he2021deberta} are pre-trained on web-scale unlabeled text corpora for general tasks such as word prediction. They are then fine-tuned for specific tasks using small amounts of labeled, task-specific data. Recent surveys on PLMs include \cite{zhou2023comprehensive}, \cite{han2021pretrained}, \cite{Qiu_2020}.

Large language models (LLMs) mainly refer to transformer-based neural language models that contain tens to hundreds of billions of parameters. These models, such as PaLM \cite{chowdhery2022palm}, LLaMA \cite{touvron2023llama}, and GPT-4 \cite{openai2024gpt4}, are pre-trained on massive text datasets. Compared to pre-trained language models (PLMs), LLMs are not only significantly larger in model size but also exhibit stronger language understanding and generation abilities. More importantly, they demonstrate emergent abilities not present in smaller-scale language models. As illustrated in Figure \ref{fig:llm-capabilities}, these emergent abilities include:

\textbf{In-context learning}: LLMs can learn a new task from a small set of examples presented in the prompt at inference time.

\textbf{Instruction following}: After instruction tuning, LLMs can follow instructions for new types of tasks without explicit examples.

\textbf{Multi-step reasoning}: LLMs can solve complex tasks by breaking them down into intermediate reasoning steps, as demonstrated in the chain-of-thought prompting \cite{wei2023chainofthought}.

LLMs can also be augmented with external knowledge and tools \cite{mialon2023augmented}, \cite{mialon2023augmented}, enabling them to effectively interact with users and the environment \cite{mialon2023augmented}, and continually improve using feedback data collected through interactions, such as via reinforcement learning with human feedback (RLHF).

\begin{figure}[H]
    \centering
    \includegraphics[width=\textwidth,height=6cm,keepaspectratio=true]{llm-capabilities.png}
    \caption{
        \it{LLM capabilities \cite{minaee2024large}}
    }
    \label{fig:llm-capabilities}
\end{figure}

\subsection{Large Language Model Families}

Large language models (LLMs) primarily refer to transformer-based pre-trained language models (PLMs) that encompass tens to hundreds of billions of parameters. Compared to the previously discussed PLMs, LLMs are significantly larger in size and demonstrate enhanced capabilities in language understanding, generation, and emergent phenomena that are absent in smaller models. This section will examine three prominent LLM families: GPT, LLaMA, and PaLM, as illustrated in Figure \ref{fig:llm-families}.

\begin{figure}[H]
    \centering
    \includegraphics[width=\textwidth,height=6cm,keepaspectratio=true]{llm-families.png}
    \caption{
        \it{Popular LLM families \cite{minaee2024large}}
    }
    \label{fig:llm-families}
\end{figure}

\textbf{The GPT Family:} Generative Pre-trained Transformers (GPT) are a series of decoder-only Transformer-based language models developed by OpenAI. This series includes models such as GPT-1, GPT-2, GPT-3, InstrucGPT, ChatGPT, GPT-4, CODEX, and WebGPT. While the earlier models, such as GPT-1 and GPT-2, are open-source, the more recent models, including GPT-3 and GPT-4, are proprietary and accessible only through APIs.

\textbf{The LLaMA Family:} LLaMA is a series of foundational language models released by Meta. In contrast to GPT models, LLaMA models are open-source, with model weights made available to the research community under a noncommercial license. Consequently, the LLaMA family has expanded rapidly, as these models are widely utilized by numerous research groups to develop improved open-source language models to compete with proprietary models or to create task-specific models for mission-critical applications.

\textbf{The PaLM Family:} The PaLM (Pathways Language Model) family, developed by Google, includes the first PaLM model \cite{chowdhery2022palm} announced in April 2022 and kept private until March 2023. This transformer-based LLM features 540 billion parameters and is pre-trained on a high-quality text corpus containing 780 billion tokens, covering a broad spectrum of natural language tasks and applications. The model's pre-training utilized 6144 TPU v4 chips within the Pathways system, facilitating highly efficient training across multiple TPU Pods. PaLM showcases the ongoing advantages of scaling, achieving state-of-the-art few-shot learning results on hundreds of language understanding and generation benchmarks. The PaLM-540B model not only surpasses state-of-the-art fine-tuned models on various multi-step reasoning tasks but also performs comparably to humans on the newly introduced BIG-bench benchmark.

\subsection{Tokenizations}

Tokenization refers to the process of converting a sequence of text into smaller units known as tokens. While the simplest tokenization tools split text based on whitespace, most rely on a word dictionary. However, this approach faces the out-of-vocabulary (OOV) problem, as the tokenizer can only recognize words present in its dictionary. To address this, popular tokenizers for LLMs are based on sub-words, which can be combined to form a vast array of words, including those not seen in the training data or from different languages. The following sections describe three popular tokenizers \cite{minaee2024large}.

\subsubsection*{BytePairEncoding}

BytePairEncoding (BPE) is originally a data compression algorithm that leverages frequent patterns at the byte level to compress data. The algorithm primarily aims to retain frequent words in their original form while breaking down less common words. This approach ensures that the vocabulary remains manageable in size while adequately representing common words. Additionally, BPE effectively represents morphological variations of frequent words if their suffixes or prefixes are commonly found in the training data.

\subsubsection*{WordPieceEncoding}

This algorithm is predominantly used in well-known models such as BERT and Electra. At the beginning of training, it incorporates the entire alphabet from the training data to ensure that no characters are left as UNK (unknown). This scenario occurs when the model encounters input that cannot be tokenized, typically involving untokenizable characters. Similar to BytePairEncoding, this approach aims to maximize the likelihood of including all tokens in the vocabulary based on their frequency.

\subsubsection*{SentencePieceEncoding}

Although the previously described tokenizers offer significant advantages over white-space tokenization, they still assume that words are always separated by white spaces. This assumption does not hold true for all languages, where words can be disrupted by various noisy elements such as unwanted spaces or invented words. SentencePieceEncoding aims to address this issue.

\subsection{Positional Encoding}

\subsubsection*{Absolute Positional Embeddings}

Absolute Positional Embeddings (APE) \cite{vaswani2023attention} has been used in the original Transformer model to preserve the sequence order information. Consequently, positional information of words is added to the input embeddings at the base of both the encoder and decoder stacks. Various options exist for positional encodings, including learned or fixed methods. In the vanilla Transformer, sine and cosine functions are utilized for this purpose. However, the main drawback of using APE in Transformers is its limitation to a fixed number of tokens. Moreover, APE does not account for the relative distances between tokens.


\subsubsection*{Relative Positional Embeddings}

Relative Positional Embeddings(RPE) \cite{shaw2018selfattention} extend self-attention to consider the pairwise connections between input elements. RPE is integrated into the model at two levels: first, as an additional component to the keys, and subsequently, as a sub-component of the values matrix. This methodology views the input as a fully-connected graph with labels and directed edges. In the context of linear sequences, edges can encode information about the relative positional disparities between input elements. A clipping distance, denoted as \( k^2 \leq k \leq n - 4 \), specifies the maximum limit on relative locations. This mechanism enables the model to make sensible predictions for sequence lengths not present in the training data.

\subsubsection*{Rotary Position Embeddings}

Rotary Positional Embedding (RoPE) \cite{su2023roformer} addresses limitations inherent in existing approaches. Learned absolute positional encodings may lack generalizability and significance, particularly in the context of short sentences. Additionally, current methods such as T5's positional embedding struggle with constructing a complete attention matrix between positions. RoPE employs a rotation matrix to encode the absolute position of words while concurrently incorporating explicit relative position details in self-attention. This approach offers several advantageous features, including flexibility with varying sentence lengths, reduced word dependency as relative distances increase, and enhancement of linear self-attention through relative position encoding. Notably, models like GPT-NeoX-20B, PaLM, CODEGEN, and LLaMA leverage RoPE within their architectures.

\subsubsection*{Relative Positional Bias}

The rationale behind this form of positional embedding is to enable extrapolation during inference for sequences longer than those encountered during training. In their work \cite{press2022train}, Press et al. introduced Attention with Linear Biases (ALiBi). Rather than solely adding positional embeddings to word embeddings, they introduced a bias to the attention scores of query-key pairs, imposing a penalty proportional to their distance. The BLOOM \cite{workshop2023bloom} model leverages ALiBi as part of its architecture.

\begin{figure}[H]
    \centering

    \begin{subfigure}[b]{.49\linewidth}
        \centering
        \includegraphics[width=\textwidth,height=6cm,keepaspectratio=true]{ape.png}
        \caption{
            \it{Absolute positional encoding \cite{ke2021rethinking}}
        }
    \end{subfigure}
    \hfill
    \begin{subfigure}[b]{.49\linewidth}
        \centering
        \includegraphics[width=\textwidth,height=6cm,keepaspectratio=true]{rpe.png}
        \caption{
            \it{Relative Positional Embeddings}
        }
    \end{subfigure}
    \par\bigskip
    \begin{subfigure}[b]{.49\linewidth}
        \centering
        \includegraphics[width=\textwidth,height=6cm,keepaspectratio=true]{alibi.png}
        \caption{
            \it{Relative Positional Bias \cite{press2022train}}
        }
    \end{subfigure}
    \hfill
    \begin{subfigure}[b]{.49\linewidth}
        \centering
        \includegraphics[width=\textwidth,height=6cm,keepaspectratio=true]{rope.png}
        \caption{
            \it{Rotary Positional Embedding \cite{su2023roformer}}
        }
    \end{subfigure}

    \caption{\it{Various positional encodings employed in LLMs}}

\end{figure}

\subsection{Model Pre-training}

Pre-training is the very first step in the large language model training pipeline, helping LLMs acquire fundamental language understanding capabilities, which can be useful in a wide range of language-related tasks. During pre-training, the LLM is trained on a massive amount of (usually) unlabeled texts, typically in a self-supervised manner. There are different approaches used for pre-training, such as next sentence prediction \cite{devlin2019bert}. The two most common techniques include next token prediction (autoregressive language modeling) and masked language modeling.

In the \textbf{Autoregressive Language Modeling} framework, given a sequence of n tokens \( x_1, \ldots, x_n \), the model aims to predict the next token \( x_{n+1} \) (and occasionally the subsequent sequence of tokens) in an autoregressive manner. A commonly used loss function in this context is the log-likelihood of the predicted tokens, as demonstrated in Equation \ref{eq:alm}.

\begin{equation}\label{eq:alm}
    \mathcal{L}_{ALM}(x) = \sum_{i=1}^N p(x_{i+n} |x_i, \ldots, x_{i+n-1})
\end{equation}

Given the autoregressive nature of this framework, decoder-only models are inherently better suited to learn and perform these tasks.

\hfill

In \textbf{Masked Language Modeling}, certain words in a sequence are masked, and the model is trained to predict these masked words based on the surrounding context. This approach is sometimes referred to as denoising autoencoding. If we denote the masked or corrupted samples in the sequence \( x \) as \( \tilde{x} \), the training objective can be expressed as:


\begin{equation}
    \mathcal{L}_{MLM}(x) = \sum_{i=1}^N p(\tilde{x}|x\backslash \tilde{x})
\end{equation}

\hfill

More recently, \textbf{Mixture of Experts (MoE)} \cite{shazeer2017outrageously}, \cite{fedus2022switch} have gained significant popularity in the LLM space. MoEs allow models to be pre-trained with substantially less compute, enabling a dramatic increase in model or dataset size within the same compute budget as a dense model. An MoE architecture consists of two main components: Sparse MoE layers, which replace dense feed-forward network (FFN) layers, and contain several "experts" (e.g., 8), each of which is a neural network. Typically, these experts are FFNs, but they can also be more complex networks. A gate network, or router, determines which tokens are sent to which expert. Notably, a token can be routed to multiple experts. Routing tokens to experts is a critical decision in MoE design; the router consists of learned parameters and is trained concurrently with the rest of the network. Figure 29 illustrates a Switch Transformer encoder block, commonly used in MoEs.


\begin{figure}[H]
    \centering
    \includegraphics[width=\textwidth,height=6cm,keepaspectratio=true]{moe.png}
    \caption{
        \it{Illustration of a Switch Transformer encoder block.
            They replaced the dense feed forward network (FFN) layer present in the Transformer with a sparse Switch FFN layer
            (light blue). \cite{fedus2022switch}}
    }
    \label{fig:moe}
\end{figure}