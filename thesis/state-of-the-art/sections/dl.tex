\section{Deep Learning}

\subsection{Definition}

Deep learning is a type of machine learning that teaches computers to perform tasks by learning from examples, much like humans do. Imagine teaching a computer to recognize cats: instead of telling it to look for whiskers, ears, and a tail, you show it thousands of pictures of cats. The computer finds the common patterns all by itself and learns how to identify a cat. This is the essence of deep learning.

In technical terms, deep learning uses something called "neural networks," which are inspired by the human brain. These networks consist of layers of interconnected nodes that process information. The more layers, the "deeper" the network, allowing it to learn more complex features and perform more sophisticated tasks \cite{datacamp:dl}.


\subsection{Deep Learning vs. Machine Learning}

Deep learning stands apart from traditional machine learning in its approach to data and learning methods.
Machine learning algorithms typically rely on structured, labeled data for predictions, where specific features are defined and organized into tables.
While machine learning can handle unstructured data, it often requires preprocessing to structure it. In contrast, deep learning streamlines this process by directly processing unstructured data such as text and images. It automates feature extraction, reducing reliance on human experts. For instance, in categorizing pet photos, deep learning algorithms autonomously identify key features, like ears, crucial for distinguishing between animals. In contrast, machine learning requires manual feature hierarchy establishment by human experts.

\subsection{Deep Learning Applications}

Deep learning has a wide range of applications across various domains due to its ability to learn complex patterns from large volumes of data. Some of the different types of applications for deep learning include:

\subsubsection*{Image Recognition and Computer Vision:}

\begin{itemize}
    \item Deep learning is extensively used for tasks such as image classification, object detection, facial recognition, and image segmentation.
    \item Applications include self-driving cars, medical image analysis, surveillance systems, and augmented reality.
\end{itemize}

\subsubsection*{Natural Language Processing (NLP):}

\begin{itemize}
    \item Deep learning is employed for understanding and generating human language, enabling tasks such as sentiment analysis, machine translation, text summarization, and chatbots.
    \item Applications include virtual assistants, language translation services, social media sentiment analysis, and customer support chatbots.
\end{itemize}


These are just a few examples of the diverse applications of deep learning, demonstrating its versatility and impact across various industries and fields.