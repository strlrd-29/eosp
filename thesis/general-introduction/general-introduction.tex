\chapter{General Introduction}

Recent advancements in natural language processing emphasize the need for systems that efficiently retrieve and generate information for complex queries. Traditional question-answering systems often struggle due to their reliance on single methods like keyword retrieval or language model generation, which can yield suboptimal outcomes. Many current systems are limited by predefined knowledge bases, hindering their adaptability and access to current internet data.

\hfill

In today's era of rapid information growth and intricate queries, traditional systems face challenges such as hallucinations and incomplete query resolution. There is a pressing demand for advanced systems that integrate diverse retrieval and generation techniques effectively.

\hfill

Our project aims to develop a Local Retrieval-Augmented Generation (RAG) agent using LLaMA3 and other open-source models. This agent will enhance answer accuracy and completeness through intelligent query routing, fallback strategies, and self-correction capabilities. It will leverage large language models and retrieval techniques to ground answers in relevant information sources, with the flexibility to resort to web searches and correct errors.

\hfill

Our approach focuses on creating a sophisticated RAG agent that seamlessly integrates document retrieval, evaluation, and generative processes. Queries undergo relevance assessment, utilizing document retrieval or web searches as necessary, followed by rigorous evaluation of generated answers for accuracy and relevance. This integration combines advanced retrieval methods with powerful generative models to enhance response quality and reliability, overcoming traditional system limitations.

\hfill

The document structure includes: the State of Play introduces BIGmama Technology, covering its mission, vision and products. The State of the Art section explores machine learning topics like deep learning, natural language processing, neural networks, transformers, large language models, and Retrieval-Augmented Generation (RAG). The Proposed Solution section defines the problem, outlines our approach and workflow, and discusses challenges and solutions. Finally, the General Conclusion summarizes key findings and implications.
