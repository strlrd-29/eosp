% \addcontentsline{toc}{section}{Abstract}
% \markboth{Abstract}{Abstract}

\chapter*{\RL{ملخص}}

\begin{RLtext}
    التطورات الأخيرة في معالجة اللغة الطبيعية سلطت الضوء على الحاجة إلى أنظمة يمكنها استرجاع وتوليد المعلومات بشكل فعال للتعامل مع الاستفسارات المعقدة بشكل متزايد. إن الجمع بين عمليتي الاسترجاع والتوليد يعالج القيود لكل منهما على حدة، مما يؤدي إلى استجابات أكثر شمولاً ودقة.

    \hfill

    تقدم هذه الأطروحة تنفيذ وكيل توليد معزز بالاسترجاع \LR{(RAG)} باستخدام \LR{Llama3} لتعزيز دقة وملاءمة الاستجابات في بيئات الاستفسارات المعقدة. التحدي الأساسي هو دمج استرجاع المعلومات الفعال مع القدرات التوليدية المتقدمة لتوفير إجابات دقيقة وموثوقة. نهجنا يجمع بين استرجاع الوثائق وتقييمها وتوليدها داخل نظام متكامل. يتم تقييم الاستفسارات للملاءمة، واسترجاع الوثائق ذات الصلة أو إجراء عمليات بحث عبر الويب حسب الحاجة. يتم تقييم الإجابات المُولدة بدقة لضمان تلبية المعايير العالية للدقة. يوضح هذا التنفيذ الإمكانية الكبيرة لدمج آليات الاسترجاع المتطورة مع النماذج التوليدية القوية، مما يؤدي إلى تحسينات كبيرة في جودة وموثوقية الاستجابات.

    \hfill

    \noindent\textbf{الكلمات المفتاحية:} معالجة اللغة الطبيعية، استرجاع المعلومات، توليد المعلومات، Llama3، استفسارات معقدة.
\end{RLtext}


\chapter*{Résumé}

Les récents progrès en traitement du langage naturel ont mis en évidence la nécessité de systèmes capables de récupérer et de générer des informations pour traiter des requêtes de plus en plus complexes. La combinaison des processus de récupération et de génération permet de pallier les limitations de chaque approche prise individuellement, aboutissant à des réponses plus complètes et précises.

\hfill

Cette thèse présente la mise en œuvre d'un agent de génération augmentée par récupération (RAG) utilisant Llama3 pour améliorer la précision et la pertinence des réponses dans des environnements de requêtes complexes. Le défi principal réside dans l'intégration d'une récupération d'informations efficace avec des capacités génératives avancées pour fournir des réponses précises et fiables. Notre approche combine récupération de documents, évaluation et génération au sein d'un système cohérent. Les requêtes sont évaluées en termes de pertinence, récupérant des documents pertinents ou effectuant des recherches sur le web si nécessaire. Les réponses générées sont rigoureusement évaluées pour garantir qu'elles répondent à des normes élevées de précision. Cette mise en œuvre démontre le potentiel de la fusion de mécanismes de récupération sophistiqués avec des modèles génératifs puissants, aboutissant à des améliorations significatives de la qualité et de la fiabilité des réponses.

\hfill

\noindent\textbf{Mots-clés:} traitement du langage naturel, récupération d'informations, génération d'informations, Llama3, requêtes complexes.


\chapter*{Abstract}

Recent advancements in natural language processing have highlighted the need for systems that can effectively retrieve and generate information to handle increasingly complex queries. Combining retrieval and generation processes addresses the limitations of each approach individually, leading to more comprehensive and accurate responses.

\hfill

This thesis presents the implementation of a Retrieval-Augmented Generation (RAG) agent utilizing Llama3 to enhance the accuracy and relevance of responses in complex query environments. The primary challenge is integrating effective information retrieval with advanced generative capabilities to provide precise and reliable answers. Our approach combines document retrieval, grading, and generation within a cohesive system. Queries are assessed for relevance, retrieving pertinent documents or conducting web searches as needed. The generated answers are rigorously evaluated to ensure they meet high standards of accuracy. This implementation demonstrates the potential of merging sophisticated retrieval mechanisms with powerful generative models, resulting in significant improvements in response quality and reliability.

\hfill

\noindent\textbf{Keywords:} natural language processing, information retrieval, information generation, Llama3, complex queries.